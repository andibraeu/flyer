\documentclass[10pt,a4paper,notumble]{leaflet}
\usepackage[T1]{fontenc}
\usepackage[ngerman]{babel}
\usepackage[utf8]{inputenc}
\usepackage{color,flowfram,graphicx}
	\graphicspath{{./bilder/}}
\usepackage{wrapfig,rotating}
\usepackage{multirow,multicol,array,mathpazo}
\usepackage{titlesec}
\usepackage{framed,libertine}
\usepackage[usenames,dvipsnames,svgnames]{xcolor}
\usepackage{rotating}

\usepackage{setspace,fontawesome} %fb and youtube icon
\usepackage[colorlinks=true, urlcolor=black]{hyperref}
\urlstyle{same}

\definecolor{ffyellow}{RGB}{255,180,0}
\definecolor{ffred}{RGB}{220,0,103}
\definecolor{ffblue}{RGB}{0,158,224}
\definecolor{ffgrey}{RGB}{51,51,51}

\titleformat*{\section}{\Large\color{ffred}}

%FONT Change
%\renewcommand{\familydefault}{cmss} 

% To draw a horizontal Line
\newcommand{\sectionline}{
 \nointerlineskip \vspace{\baselineskip}
 \hspace{\fill}\rule{0.8\linewidth}{.7pt}\hspace{\fill}
 \par\nointerlineskip \vspace{\baselineskip}
}

% Add graphics to bottom of the first page
%\AddToBackground{2}{\includegraphics[width=70mm]{weimarnetz_logo.png}}

% Make a border along the top of each page
\setmargins{15mm}{15mm}{7mm}{7mm}
\vtwotonetop{1cm}{0.6\paperwidth}{ffyellow}{topleft}{0.4\paperwidth}{ffyellow}{topright}
\vtwotonebottom[1,6]{1cm}{0.6\paperwidth}{ffyellow}{bottomleft}{0.4\paperwidth}{ffyellow}{bottomright}

\pagestyle{empty}

\begin{document}

% \begin{titlepage}
% \title{Teilt Eure WLANs!}
% \date{}
% \end{titlepage}

% \maketitle

\begin{center}
{\fontsize{70}{90}\selectfont \faWifi}
\end{center}
\begin{center}
{\fontsize{30}{40}\selectfont Teilt\ Eure WLANs!}
\end{center}

\vfill

{\fontsize{16}{18}\selectfont Tausende weimarer Schüler:innen befinden sich im Fernunterricht. Dank Corona ist die Digitalisierung quasi über Nacht auch in den Schulen angekommen. Doch nicht alle Schüler:innen haben zu Hause einen Internetzugang. Mit Freifunk kannst Du helfen und Deinen Internetzugang einfach und sicher teilen.\par}

\vspace{1em}

\begin{center}
\includegraphics[width=70mm]{weimarnetz_logo.png}
\end{center}


\thispagestyle{empty}
\newpage

\section{Was ist Freifunk?}
Freifunk ist eine nichtkommerzielle Initiative für freie Funk\-netzwerke. Wir bauen ein Gemeinschaftsnetz und stellen anderen unsere Internetzugänge zur Verfügung. In diesem Flyer erfährst Du wie Du mitmachen kannst.

Du brauchst keine besonderen technischen Kenntnisse, um Teil des Netzes zu werden. Es genügt, 1. ein passendes Gerät -- einen herkömmlichen WLAN-Router, den sog. Freifunk-Knoten -- zu kaufen, 2. die Freifunk-Firmware nach Anleitung aufzuspielen, 3. den Knoten einzurichten und 4. anzuschließen.

% Alternativ kannst Du bei TODO ein fertig eingerichtetes Gerät abholen, das nur noch angeschlossen werden muss.

\section{1. Gerät kaufen}
Je nach Geldbeutel und Installationsort eignen sich verschiedene Geräte. Wir empfehlen an dieser Stelle zwei Geräte, die leicht eingerichtet werden können: den \textbf{TP-Link C6 v2} (ca. 40€) und den \textbf{TP-Link C60 v2} (ca. 45€).

Grundsätzlich sind alle Geräte geeignet, die unter\\\href{https://www.weimarnetz.de/category/router}{https://www.weimarnetz.de/category/router} aufgeführt werden. Manche der dortigen Geräte gibt es in verschiedenen Hardware-Revisionen. Bitte achtet beim Kauf darauf. Nicht immer werden alle Revisionen von uns unterstützt.

Sobald ein passendes Gerät beschafft wurde, kann die Firmware aufgespielt werden.

\section{2. Firmware aufspielen}

\paragraph{TP-Link Geräte}
\begin{enumerate}
 \item Lade Dir zunächst die Erstinstallations-Firmware für Dein Gerät herunter:\\\href{https://weimarnetz.de/router/archer-c6}{https://weimarnetz.de/router/archer-c6}
 \item Schließ den Router ans Stromnetz an. Warte bis die Power- und die WiFi-LED durchgehend eingeschaltet sind.
 \item Verbinde jetzt einen gelben Switchport mit dem Netzwerkanschluss Deines Computers und den blauen WAN-Port mit deinem Internetrouter.
 \item Öffne in einem Browserfenster die Adresse \textit{http://tplinkwifi.net/} oder \textit{http://192.168.0.1} und melde Dich mit \textit{admin} und dem Paßwort \textit{admin} an.
 \item Klicke links im Menü auf \textit{System Tools} und dann im Reiter \textit{Firmware Upgrades} auf \textit{Manual Upgrade}.
 \item Wähle die Freifunk-Firmware über den Button \textit{Browse}, die Du gerade von unserer Seite runtergeladen hast. Klicke anschließend auf \textit{Upgrade}.
 \item Nun warte etwas bis das Upgrade durchgelaufen ist und sich der Router selbstständig neu startet.
\end{enumerate}


%\paragraph{Gl.iNet GL-B1300}
%\begin{enumerate}
% \item Lade zunächst die Firmware für Dein Gerät herunter: \href{https://hamburg.freifunk.net/firmware}{https://hamburg.freifunk.net/firmware}
% \item Schließe den Router ans Stromnetz an. Verbinde dann den \textit{mittleren LAN-Port} mit dem Netzwerkanschluss des Computers.
% \item Öffne in einem Browserfenster die Adresse \textit{http://192.168.8.1}
% \item Wähle nun eine Sprache und vergebe ein Passwort. Klicke danach links auf \textit{Aktualisierung} und dann auf den Reiter \textit{lokales Upgrade}.
% \item Wähle die Freifunk-Firmware aus, \textbf{deaktiviere} \textit{Einstellung behalten} und klicke anschließend auf \textit{Installieren}.
% \item Nun warte etwa drei Minuten bis das Upgrade durchgelaufen ist. Danach kann der Knoten eingerichtet werden.
%\end{enumerate}


%\newpage

\section{3. Knoten einrichten}
Danach richtet sich der Knoten selbstständig ein. Dieser Vorgang kann bis zu einer Stunde dauern.

Damit ist die Registrierung abgeschlossen und Du kannst Deinen Freifunk-Router in Betrieb nehmen.

Du solltest ein neues, offenes WLAN mit der SSID ''\mbox{\textit{Weimarnetz.Freifunk}}'' oder ''\mbox{\textit{weimarnetz.de | XXX}}'' sehen.\\
''XXX'' steht hier für die Knotennumer Deines neuen Routers. Dieser wurde bei der Einrichtung automatisch vergeben.
\\\\
Teste es am besten kurz und sage dann Deinen Nachbarn Bescheid!  

\begin{center}
\includegraphics[width=20mm]{Weimarnetz.Freifunk.qr.png}\\
\footnotesize Weimarnetz.Freifunk
\end{center}

\newpage


\vspace{1em}

\begin{center}
\includegraphics[width=40mm]{weimarnetz_logo.png}
\end{center}
\begin{flushright}
Stand: 04.2021
\end{flushright}


\newpage
\section{Häufige Fragen}
\setlength{\parskip}{0.1em}
\paragraph{Wo stehen schon Freifunk-Router?}
Eine Karte findest Du unter \href{https://weimarnetz.de/knotenkarte}{https://weimarnetz.de/knotenkarte}.

\paragraph{Was kostet Freifunk?}
Für die Nutzung fallen für Dich keine Kosten an. Die Infrastruktur wird ausschließlich über Spenden finanziert.

Falls Du selbst einen Freifunk-Zugang anbieten möchtest, benötigst Du einen eigenen Internetanschluss und einen passenden Router. Dieser kostet je nach Modell ab ca. 35€.  Neben den Stromkosten für das Betreiben des Routers fallen keine weitere Kosten an.

\paragraph{Wie weit reicht ein Freifunk-Knoten?}
Freifunk nutzt ganz normales WLAN. Bei optimalen Bedingungen reicht das WLAN mehr als 100m weit. Jeder Gegenstand (Wand, Fenster, Baum, Mensch…) wirkt jedoch dämpfend auf das Signal. Praktisch reicht es oft zumindest bis in die Nachbarwohnung.

\paragraph{Ist das sicher?}
Man kann aus dem Freifunk-Netz nicht auf Dein privates Netz zugreifen.

Bei der Nutzung herrschen die gleichen Bedingungen wie in jedem anderen WLAN: Die Nutzer:innnen müssen verantwortlich damit umgehen. Das bedeutet u.a.: die eigenen Endgeräte (Handy, Laptop etc.) mit den üblichen Vorkehrungen vor Fremdzugriff zu schützen, Verschlüsselung \& https zu benutzen sowie regelmäßige Updates der Router und Endgeräte durchzuführen.

\paragraph{Wo kann ich weitere Informationen finden?}
Auf unserer Webseite \href{https://weimarnetz.de}{https://weimarnetz.de} und auf den Seiten des ''Förderverein Freie Netzwerke e. V.'' unter \\\href{https://freifunk.net}{https://freifunk.net}.

\newpage
\begin{center}
\includegraphics[width=65mm]{weimarnetz_logo.png}
\end{center}
\vspace{3em}
\begin{center}
\includegraphics[width=40mm]{qr.png}\\
\footnotesize Dieser Flyer zum Download: https://weimarnetz.de/infoflyer
\end{center}
\vfill
\section{Kontakt}
\begin{tabular}{ll}
WWW & \href{https://weimarnetz.de}{https://weimarnetz.de}\\
E-Mail & \href{mailto:kontakt@weimarnetz.de}{kontakt@weimarnetz.de}\\
        \\
Treffen & dienstags, online\\
	& \href{https://meet.weimarnetz.de/ImMaschinenraum}{\small https://meet.weimarnetz.de/ImMaschinenraum}
\end{tabular}

\end{document}
